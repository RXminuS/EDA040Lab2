\documentclass{article}
\usepackage{lipsum}
\usepackage[margin=1.25in,letterpaper]{geometry}
\usepackage{listings}

\textbackslash{}begin\{document\}

\section{AutoComp Documentation}

\subsection{Imports:}

\begin{Shaded}
\begin{Highlighting}[]
\KeywordTok{module} \DataTypeTok{AutoComp} \KeywordTok{where}
\KeywordTok{import }\DataTypeTok{Haskore} \KeywordTok{hiding} \NormalTok{(chord, }\DataTypeTok{Key}\NormalTok{)}
\KeywordTok{import }\DataTypeTok{Data.List}
\KeywordTok{import }\DataTypeTok{Data.Maybe}
\end{Highlighting}
\end{Shaded}

\subsection{Data Structures:}

A key (consisting of a pitch class like C, D or E and a Major/Minor
mode) denotes the key the song is played in

\begin{Shaded}
\begin{Highlighting}[]
\KeywordTok{type} \DataTypeTok{Key} \FunctionTok{=} \NormalTok{(}\DataTypeTok{PitchClass}\NormalTok{, }\DataTypeTok{Mode}\NormalTok{)}
\end{Highlighting}
\end{Shaded}

Although many more `chord' triads exist we've restricted ourselves to a
Major and Minor chord triad. These structures are used to find the right
notes for a particular chord voicing.

\begin{Shaded}
\begin{Highlighting}[]
\KeywordTok{data} \DataTypeTok{Triad} \FunctionTok{=} \DataTypeTok{TriMaj} \FunctionTok{|} \DataTypeTok{TriMin} \KeywordTok{deriving} \NormalTok{(}\DataTypeTok{Eq}\NormalTok{)}
\end{Highlighting}
\end{Shaded}

In a chord the PitchClass denotes the root note and the Triad will
change the accompanying notes or chord voicing.

\begin{Shaded}
\begin{Highlighting}[]
\KeywordTok{type} \DataTypeTok{Chord} \FunctionTok{=} \NormalTok{(}\DataTypeTok{PitchClass}\NormalTok{, }\DataTypeTok{Triad}\NormalTok{)}
\end{Highlighting}
\end{Shaded}

Our bass style makes us swing in different ways.

\begin{Shaded}
\begin{Highlighting}[]
\KeywordTok{data} \DataTypeTok{BassStyle} \FunctionTok{=} \DataTypeTok{Basic} \FunctionTok{|} \DataTypeTok{Calypso} \FunctionTok{|} \DataTypeTok{Boogie} \KeywordTok{deriving} \NormalTok{(}\DataTypeTok{Eq}\NormalTok{)}
\end{Highlighting}
\end{Shaded}

A chord progression is used to describe a song where each Chord is
played for Dur duration

\begin{Shaded}
\begin{Highlighting}[]
\KeywordTok{type} \DataTypeTok{ChordProgression} \FunctionTok{=} \NormalTok{[(}\DataTypeTok{Chord}\NormalTok{,}\DataTypeTok{Dur}\NormalTok{)]}
\end{Highlighting}
\end{Shaded}

A ChordPitch is a more specific representation of a chord when compared
to a general Chord data type. It explicitly states the pitches of all
three notes (we only have triad chords) and is used as an intermediary
step between a general Chord and a final Music data structure.

\begin{Shaded}
\begin{Highlighting}[]
\KeywordTok{type} \DataTypeTok{ChordPitch} \FunctionTok{=} \NormalTok{(}\DataTypeTok{Pitch}\NormalTok{, }\DataTypeTok{Pitch}\NormalTok{, }\DataTypeTok{Pitch}\NormalTok{)}
\end{Highlighting}
\end{Shaded}

\subsection{Helper Functions:}

This function was skilfully copied from the supplied code and thus
requires no further explanation. The only adjustment that has been made
is a reduced volume to enhance the final result.

\begin{Shaded}
\begin{Highlighting}[]
\NormalTok{fd d n }\FunctionTok{=} \NormalTok{n d v}
\NormalTok{vol  n }\FunctionTok{=} \NormalTok{n   v}
\NormalTok{v      }\FunctionTok{=} \NormalTok{[}\DataTypeTok{Volume} \DecValTok{60}\NormalTok{]}
\end{Highlighting}
\end{Shaded}

Again another beautiful example of `better stolen well than thought of
poorly'. This function is maily used to explicitly state repetition
instead of implicitly repeating chords.

\begin{Shaded}
\begin{Highlighting}[]
\CommentTok{-- repeat something n times}
\NormalTok{times }\DecValTok{1}     \NormalTok{m }\FunctionTok{=} \NormalTok{m}
\NormalTok{times n m }\FunctionTok{=} \NormalTok{m }\FunctionTok{:+:} \NormalTok{(times (n }\FunctionTok{-} \DecValTok{1}\NormalTok{) m)}
\end{Highlighting}
\end{Shaded}

Although small shift is used throughout both the bass line and chord
voicing generation. It's role is to iteratively append the head to the
tail (removing the head) 0..inf times. We then simply pick the n-th
iteration and due to Haskells lazy evaluation this should be the only
iteration actually calculated. Thank you Haskell!

\begin{Shaded}
\begin{Highlighting}[]
\OtherTok{shift ::} \DataTypeTok{Eq} \NormalTok{a }\OtherTok{=>} \DataTypeTok{Int} \OtherTok{->} \NormalTok{[a] }\OtherTok{->} \NormalTok{[a]}
\NormalTok{shift n l}\FunctionTok{=} \NormalTok{(iterate f l)}\FunctionTok{!!}\NormalTok{n}
      \KeywordTok{where} \NormalTok{f [] }\FunctionTok{=} \NormalTok{[]}
            \NormalTok{f (x}\FunctionTok{:}\NormalTok{xs) }\FunctionTok{=} \NormalTok{xs }\FunctionTok{++} \NormalTok{[x]}
\end{Highlighting}
\end{Shaded}

\subsection{Bass Line:}

The scalePattern returns all the notes for a particular key. For
instance a C Major key is made up of CDEFGAB where each note is obtained
by transposing the root note. Since this function is only used for
generating the baseline we hardcoded a root note in the 3rd octave.

\begin{Shaded}
\begin{Highlighting}[]
\OtherTok{scalePattern ::} \DataTypeTok{Key} \OtherTok{->}\NormalTok{[}\DataTypeTok{PitchClass}\NormalTok{]}
\NormalTok{scalePattern (pc, }\DataTypeTok{Major}\NormalTok{) }\FunctionTok{=} \NormalTok{fst}\FunctionTok{.}\NormalTok{unzip}\FunctionTok{.}\NormalTok{zipWith trans [}\DecValTok{0}\NormalTok{, }\DecValTok{2}\NormalTok{, }\DecValTok{4}\NormalTok{, }\DecValTok{5}\NormalTok{, }\DecValTok{7}\NormalTok{, }\DecValTok{9}\NormalTok{, }\DecValTok{11}\NormalTok{] }\FunctionTok{$} \NormalTok{repeat (pc, }\DecValTok{3}\NormalTok{)}
\NormalTok{scalePattern (pc, }\DataTypeTok{Minor}\NormalTok{) }\FunctionTok{=} \NormalTok{fst}\FunctionTok{.}\NormalTok{unzip}\FunctionTok{.}\NormalTok{zipWith trans [}\DecValTok{0}\NormalTok{, }\DecValTok{2}\NormalTok{, }\DecValTok{3}\NormalTok{, }\DecValTok{5}\NormalTok{, }\DecValTok{7}\NormalTok{, }\DecValTok{8}\NormalTok{, }\DecValTok{10}\NormalTok{] }\FunctionTok{$} \NormalTok{repeat (pc, }\DecValTok{3}\NormalTok{)}
\end{Highlighting}
\end{Shaded}

The chordMode function is used to generate a list of fitting notes based
on the key of the song and a particular chord. This daunting looking
function achieves it's greatness in a very simple manner. In the first
step we shift the scale pattern (of the song key) to the root note of
the chord. For instance the scale C3-D3-E3-F3-G3-A3-B3 when applied to a
GMaj chord becomes G3-A3-B3-C3-D3-E3-F3. Notice that that the pitches
aren't increasing correctly since we simply shift and aren't transposing
the notes. Therefore the octaveSplit and cleanOctave functions are used
to first identify where this indescrepency happens and then to fix it.
So G3-A3-B3-C3-D3-E3-F3 finally becomes G3-A3-B3-C4-D4-E4-F4.

\begin{Shaded}
\begin{Highlighting}[]
\OtherTok{chordMode ::} \DataTypeTok{Key} \OtherTok{->} \DataTypeTok{Chord} \OtherTok{->} \NormalTok{[}\DataTypeTok{Dur}\OtherTok{->}\NormalTok{[}\DataTypeTok{NoteAttribute}\NormalTok{]}\OtherTok{->} \DataTypeTok{Music}\NormalTok{]}
\NormalTok{chordMode key (pc,_) }\FunctionTok{=} \NormalTok{map }\DataTypeTok{Note} \FunctionTok{$} \NormalTok{cleanOctave }\FunctionTok{$} \NormalTok{zip transformedPitches (repeat }\DecValTok{3}\NormalTok{)}
                       \KeywordTok{where} \NormalTok{f (}\DataTypeTok{Just} \NormalTok{a) }\FunctionTok{=} \NormalTok{shift a (scalePattern key)}
                             \NormalTok{f _  }\FunctionTok{=} \NormalTok{take }\DecValTok{12} \FunctionTok{$} \NormalTok{repeat pc}
                             \NormalTok{transformedPitches }\FunctionTok{=} \NormalTok{f (elemIndex pc (scalePattern key))}
                             \NormalTok{octaveSplit scale }\FunctionTok{=} \NormalTok{partition (octaveSplitTest scale) scale}
                             \NormalTok{octaveSplitTest scale p }\FunctionTok{=} \NormalTok{absPitch p }\FunctionTok{>=} \NormalTok{(head }\FunctionTok{$} \NormalTok{map absPitch scale)}
                             \NormalTok{cleanOctave scale }\FunctionTok{=} \NormalTok{(fst}\FunctionTok{.}\NormalTok{octaveSplit) scale }\FunctionTok{++} \NormalTok{map (trans }\DecValTok{12}\NormalTok{) (snd}\FunctionTok{.}\NormalTok{octaveSplit }\FunctionTok{$} \NormalTok{scale)}
\end{Highlighting}
\end{Shaded}

The autoBass function is where the low-register magic happens. First the
key of the Chord is transformed to fit the key of the song generating a
`list of fitting notes to pick from'. By performing pattern matching on
the BassStyle for each Chord in the ChordProgression a unique picking
pattern is generated. The picking pattern consists of the semitone
differences between the root note. This picking pattern further depends
on the duration of the chord where the only cases handled are where the
chord is a `wn' (Whole Note) or a `hn' (Half Note).

\begin{Shaded}
\begin{Highlighting}[]
\OtherTok{autoBass ::} \DataTypeTok{BassStyle}\OtherTok{->} \DataTypeTok{Key} \OtherTok{->} \DataTypeTok{ChordProgression} \OtherTok{->} \DataTypeTok{Music}
\NormalTok{autoBass style key [] }\FunctionTok{=} \DataTypeTok{Rest} \DecValTok{0}
\NormalTok{autoBass }\DataTypeTok{Basic} \NormalTok{key (chord}\FunctionTok{:}\NormalTok{[])}
  \FunctionTok{|} \NormalTok{(snd chord }\FunctionTok{==} \NormalTok{hn) }\FunctionTok{=} \NormalTok{line }\FunctionTok{$} \NormalTok{zipWith fd [hn] [t}\FunctionTok{!!}\DecValTok{0}\NormalTok{]}
  \FunctionTok{|} \NormalTok{(snd chord }\FunctionTok{==} \NormalTok{wn) }\FunctionTok{=} \NormalTok{line }\FunctionTok{$} \NormalTok{zipWith fd [hn,hn] [t}\FunctionTok{!!}\DecValTok{0}\NormalTok{,t}\FunctionTok{!!}\DecValTok{4}\NormalTok{]}
  \FunctionTok{|} \NormalTok{otherwise }\FunctionTok{=} \DataTypeTok{Rest} \NormalTok{(snd chord)}
  \KeywordTok{where} \NormalTok{t }\FunctionTok{=}  \NormalTok{chordMode key (fst }\FunctionTok{$} \NormalTok{chord)}

\NormalTok{autoBass }\DataTypeTok{Calypso} \NormalTok{key (chord}\FunctionTok{:}\NormalTok{[])}
  \FunctionTok{|} \NormalTok{(snd chord }\FunctionTok{==} \NormalTok{hn) }\FunctionTok{=} \NormalTok{bar}
  \FunctionTok{|} \NormalTok{(snd chord }\FunctionTok{==} \NormalTok{wn) }\FunctionTok{=} \NormalTok{times }\DecValTok{2} \NormalTok{bar}
  \FunctionTok{|} \NormalTok{otherwise }\FunctionTok{=} \DataTypeTok{Rest} \NormalTok{(snd chord)}
  \KeywordTok{where} \NormalTok{bar }\FunctionTok{=} \DataTypeTok{Rest} \NormalTok{qn }\FunctionTok{:+:} \NormalTok{(line }\FunctionTok{$} \NormalTok{zipWith fd [en,en] [t}\FunctionTok{!!}\DecValTok{0}\NormalTok{,t}\FunctionTok{!!}\DecValTok{2}\NormalTok{])}
        \NormalTok{t   }\FunctionTok{=} \NormalTok{chordMode key (fst }\FunctionTok{$} \NormalTok{chord)}

\NormalTok{autoBass }\DataTypeTok{Boogie} \NormalTok{key (chord}\FunctionTok{:}\NormalTok{[])}
  \FunctionTok{|} \NormalTok{(snd chord }\FunctionTok{==} \NormalTok{hn) }\FunctionTok{=} \NormalTok{bar}
  \FunctionTok{|} \NormalTok{(snd chord }\FunctionTok{==} \NormalTok{wn) }\FunctionTok{=} \NormalTok{times }\DecValTok{2} \NormalTok{bar}
  \FunctionTok{|} \NormalTok{otherwise }\FunctionTok{=} \DataTypeTok{Rest} \NormalTok{(snd chord)}
  \KeywordTok{where} \NormalTok{bar }\FunctionTok{=} \NormalTok{line }\FunctionTok{$} \NormalTok{zipWith fd [en,en,en,en] [t}\FunctionTok{!!}\DecValTok{0}\NormalTok{,t}\FunctionTok{!!}\DecValTok{4}\NormalTok{,t}\FunctionTok{!!}\DecValTok{5}\NormalTok{,t}\FunctionTok{!!}\DecValTok{4}\NormalTok{]}
        \NormalTok{t }\FunctionTok{=}  \NormalTok{chordMode key (fst }\FunctionTok{$} \NormalTok{chord)}
\NormalTok{autoBass style key (c}\FunctionTok{:}\NormalTok{cs) }\FunctionTok{=} \NormalTok{autoBass style key [c] }\FunctionTok{:+:} \NormalTok{autoBass style key cs}
\end{Highlighting}
\end{Shaded}

\subsection{Chord Voicing:}

The toChord function transforms a ChordPitch into music by applying a
specific duration and transforming each individual Pitch into a
corresponding Note. Since the ChordPitch is a much more appropriate
datatype than Music to use during calculations this function bridges the
abstraction layer it imposes.

\begin{Shaded}
\begin{Highlighting}[]
\OtherTok{toChord ::} \DataTypeTok{Dur} \OtherTok{->} \DataTypeTok{ChordPitch} \OtherTok{->} \DataTypeTok{Music}
\NormalTok{toChord dur (a,b,c) }\FunctionTok{=} \NormalTok{foldl (\textbackslash{}acc x }\OtherTok{->} \NormalTok{acc }\FunctionTok{:=:} \NormalTok{(fd dur }\FunctionTok{$} \DataTypeTok{Note} \NormalTok{x) ) (}\DataTypeTok{Rest} \DecValTok{0}\NormalTok{) [a,b,c]}
\end{Highlighting}
\end{Shaded}

This operator (notice the fitting notation) takes the absolute value of
the difference between two pitches. For instance
\(C,3 |-| C,4 = C,4 |-| C,3 = 12\).

\begin{Shaded}
\begin{Highlighting}[]
\OtherTok{(|-|) ::} \DataTypeTok{Pitch} \OtherTok{->} \DataTypeTok{Pitch} \OtherTok{->} \DataTypeTok{Int}
\NormalTok{(}\FunctionTok{|-|}\NormalTok{) a b }\FunctionTok{=} \NormalTok{abs (absPitch a }\FunctionTok{-} \NormalTok{absPitch b)}
\end{Highlighting}
\end{Shaded}

The distance function returns the sum of the distance between a chords
corresponding notes. Since we always represent chords in a uniform
matter (lowest notes to the left) this function returns a correct value.

\begin{Shaded}
\begin{Highlighting}[]
\OtherTok{distance ::} \DataTypeTok{ChordPitch} \OtherTok{->} \DataTypeTok{ChordPitch} \OtherTok{->} \DataTypeTok{Int}
\NormalTok{distance (a1, b1, c1) (a2, b2, c2) }\FunctionTok{=} \NormalTok{(a1}\FunctionTok{|-|}\NormalTok{a2) }\FunctionTok{+} \NormalTok{(b1}\FunctionTok{|-|}\NormalTok{b2) }\FunctionTok{+} \NormalTok{(c1}\FunctionTok{|-|}\NormalTok{c2)}
\end{Highlighting}
\end{Shaded}

This nifty little minimize function picks a chord from a list of chords
where the distance (as defined previously) between it and another
supplied `base-chord' is minimized.

\begin{Shaded}
\begin{Highlighting}[]
\OtherTok{minimize ::} \DataTypeTok{ChordPitch} \OtherTok{->} \NormalTok{[}\DataTypeTok{ChordPitch}\NormalTok{] }\OtherTok{->} \DataTypeTok{ChordPitch}
\NormalTok{minimize prev (x}\FunctionTok{:}\NormalTok{xs) }\FunctionTok{=} \NormalTok{foldl (\textbackslash{}acc y }\OtherTok{->} \KeywordTok{if} \NormalTok{distance prev y }\FunctionTok{<} \NormalTok{distance prev acc }\KeywordTok{then} \NormalTok{y }\KeywordTok{else} \NormalTok{acc) x xs}
\end{Highlighting}
\end{Shaded}

The permutate function returns all the possible chord voicings within a
limited range of pitches. Since this function is only used for chord
voicing the root note of is fixed to the 4th octave. It enforce that the
first rule in the lab-requirements is respected. The third rule is
respected also because the hardcoded patterns always generate the
closest voicing within the triad.

\begin{Shaded}
\begin{Highlighting}[]
\OtherTok{permutate ::} \DataTypeTok{Chord} \OtherTok{->} \NormalTok{[}\DataTypeTok{ChordPitch}\NormalTok{]}
\NormalTok{permutate (p, t)}
          \FunctionTok{|} \NormalTok{t }\FunctionTok{==} \DataTypeTok{TriMin} \FunctionTok{=} \NormalTok{map toTuple (zipWith (zipWith trans) minPat rootNote)}
          \FunctionTok{|} \NormalTok{otherwise }\FunctionTok{=} \NormalTok{map toTuple (zipWith (zipWith trans) majPat rootNote)}
          \KeywordTok{where} \NormalTok{toTuple [a,b,c] }\FunctionTok{=} \NormalTok{(a,b,c)}
                \NormalTok{minPat }\FunctionTok{=} \NormalTok{[[}\DecValTok{0}\NormalTok{,}\DecValTok{3}\NormalTok{,}\DecValTok{7}\NormalTok{],[}\DecValTok{3}\NormalTok{,}\DecValTok{7}\NormalTok{,}\DecValTok{12}\NormalTok{],[}\DecValTok{7}\NormalTok{,}\DecValTok{12}\NormalTok{,}\DecValTok{15}\NormalTok{]]}
                \NormalTok{majPat }\FunctionTok{=} \NormalTok{[[}\DecValTok{0}\NormalTok{,}\DecValTok{4}\NormalTok{,}\DecValTok{7}\NormalTok{],[}\DecValTok{4}\NormalTok{,}\DecValTok{7}\NormalTok{,}\DecValTok{12}\NormalTok{],[}\DecValTok{7}\NormalTok{,}\DecValTok{12}\NormalTok{,}\DecValTok{16}\NormalTok{]]}
                \NormalTok{rootNote }\FunctionTok{=} \NormalTok{(repeat}\FunctionTok{.}\NormalTok{repeat) (p,}\DecValTok{4}\NormalTok{)}
\end{Highlighting}
\end{Shaded}

Finaly the autoChord function combines the power of `minimize' and
permutate by for each chord in the ChordProgression picking the voicing
that lies closest to the previously played chord (just guaranteeing rule
2). The first chord played will always have its root in the lowest note
since we select the first item from all available permutations. The
combine helper function recusively applies this chord selection
algorithm to the end of the song and then the main foldl uses the chosen
chords to transform into the final Music.

\begin{lstlisting}[breaklines,basicstyle=\ttfamily]
> autoChord :: Key -> ChordProgression -> Music
> autoChord _ cp@((c,_):_) = foldl (\acc x -> acc :+: x) (Rest 0) (combine ((permutate c)!!0) cp)
>                          where combine last ((c,d):[]) = toChord d (minimize last (permutate c)):[]
>                                combine last ((c,d):cp) = (toChord d (minimize last (permutate c)) ):(combine (minimize last (permutate c)) cp)
\end{lstlisting}

\subsection{Final Thoughts}

We haven't implemented the autoComp function because during the
experiments with different volumes and instruments we found that we
needed the freedom to choose these individually for the bass and chord
lines. Therefore we chose to simply generate and combine these lines in
the music files `twinkle.hs' and `clocks.hs'.

\textbackslash{}end\{document\}
